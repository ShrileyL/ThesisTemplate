% Document Structure
%% declare  %% preamble  %% airticle

%%%%%%%%文档类声明 %%%%%%%%%%%%%%
%article、report、book
\documentclass[a4paper,12pt,draft]{report} 

%%%%%%%%%%%宏包%%%%%%%%%%%%%%%%
%\usepackage[options]{package} %引入宏包 ...

\usepackage{fontspec,xltxtra,xunicode} %最新的mactex都有
\usepackage[normalem]{ulem}%字体样式,下划线宏包
\usepackage{marginnote} %边注宏包

%%%%%%%%%%%%%%%%%%%%%%%%%%%%%

\defaultfontfeatures{Mapping=tex-text}
\setromanfont{Heiti SC} %设置中文字体
\XeTeXlinebreaklocale “zh”
\XeTeXlinebreakskip = 0pt plus 1pt minus 0.1pt %文章内中文自动换行,可以自行调节

\newfontfamily{\H}{Songti SC} %设定新的字体快捷命令
\newfontfamily{\E}{Arial}%设定新的字体快捷命令

\renewcommand{\thefootnote}{\roman{footnote}} %i, ii, iii


%%%%%%%%%%%%%正文%%%%%%%%%%%%%%%
\begin{document} 
\title{LaTeX Notes} 
\author{Shirley} 
\date{\today} 
\maketitle

 \begin{abstract}% book 中没有abstract层次结构
abstract
\end{abstract}
 
  \setcounter{tocdepth}{2} % 设 定 目 录 深 度 
   \tableofcontents   % 列 出 目 录

 \chapter{入门}% airticle 中没有chapter层次结构
 
 \section{常用命令环境}
  \subsection{字体和下划线}
 \emph{emphasis}\\ 
 \uline{underline}\\ 
 \uwave{waveline}\\ 
 \sout{strike-out}
 
 \subsection{列表}
 %\subsection{}
 %\subsubsection{}

%列表
无序列表:
\begin{itemize}
	\item C++
	\item Java
	\item HTML
\end{itemize}

有序列表:
\begin{enumerate}
	\item C++
	\item Java
	\item HTML
\end{enumerate}

描述列表:
\begin{description} 
	\item{C++} 一种编程语言 
	\item{Java} 另一种编程语言 
	\item{HTML} 一种标记语言
\end{description}

 \subsection{对齐}
 \begin{flushleft} 
 本段落\\
 居左 
 \end{flushleft}
 
  \begin{flushright} 
  本段落\\
  居右 
  \end{flushright}
 
 \begin{center} 
 本段落\\
 居中 
 \end{center}
 
\subsection{摘录}
正文
\begin{quote}
引文两端都缩进。
\end{quote}
正文

正文
\begin{quotation}
引文两端缩进,首行缩进。
\end{quotation}
正文

正文
\begin{verse}
引文两端缩进,第二行起缩进。
\end{verse}
正文

\subsection{脚注}
%重定义脚注,在document之前
%\renewcommand{\thefootnote}{\roman{footnote}} %i, ii, iii
正文\footnote{脚注}

\subsection{边注}
\marginnote{正常边注} \reversemarginpar
\marginnote{反向边注} \normalmarginpar

\end{document}






%hello_world.tex

%Example #1
%\documentclass{article}
%\begin{document}
%    Hello, World!
%\end{document}

%Example #2
%cannot complied sucessiful

%\documentclass{article}
%\usepackage{CJK}
%\usepackage{indentfirst}
%\setlength{\parindent}{2em}
%\usepackage{geometry}
%\usepackage{graphicx}
%%\geometry{left=1.0cm,right=1.0cm,top=1.0cm,bottom=1.0}
%\begin{document}
%\begin{CJK*}{GBK}{song}
%hello
%你好吗
%\end{CJK*}
%\end{document}

%Example #3
%\documentclass{article}
%\begin{document}
%\title{How to Print it}
%\author{aqua2001}
%\date{2012.5.20}
%\maketitle
%%0Š10—60‚10‰90—00…%4šŠ0ƒ90„40‰90“3000Š10—6000‰90—00‚5šš0„30‚20Š30ƒ40…%40Š30…%20„50‰90“4020Š30ƒ2070‡00„2
%Hello World!
%\end{document}

%Example #4
%\documentclass{article}
%\begin{document}
%\title{How to Print it}
%\author{aqua2001}
%\date{2012.5.20}
%\maketitle
%\section{The first point}
%Hello world!
%\section{The second point}
%What is matrix? and so on.
%\end{document}

%Example #5
