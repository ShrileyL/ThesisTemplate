%%%%%%%%%%%%%%%%%%%%%%%%%%%%%%%%%%%%%%%%%%%%%
%%%%%%%%%%%%%%%%%%%%%%
% !Mode:: "TeX:UTF-8"
% filename: temp_xeCJK.tex
\documentclass[8pt,a4paper,openany,twoside]{book}
%\documentclass[8pt,a4paper,openany,twoside]{article}
\usepackage{fontspec,xltxtra,xunicode}
\usepackage[slantfont,boldfont]{xeCJK}
 
% 设置中文字体
% ==========================================================
\setCJKmainfont[BoldFont=STHeiti,ItalicFont=STKaiti]{STSong}
\setCJKsansfont{STHeiti}
\setCJKmonofont{STFangsong}
 
\setCJKfamilyfont{zhsong}{STSong}
\setCJKfamilyfont{zhhei}{STHeiti}
\setCJKfamilyfont{zhfs}{STFangsong}
\setCJKfamilyfont{zhkai}{STKaiti}
 
\newcommand*{\songti}{\CJKfamily{zhsong}} % 宋体
\newcommand*{\heiti}{\CJKfamily{zhhei}} % 黑体
\newcommand*{\kaishu}{\CJKfamily{zhkai}} % 楷书
\newcommand*{\fangsong}{\CJKfamily{zhfs}} % 仿宋
% ==========================================================
 
\title{Photometric redshifts estimate from sed fitting}
\author{TangChao}
\date{\today}
 
\begin{document}
\maketitle
 
在这篇文章中,我们研究了由标准sed拟合方法获得的红移精度问题,其中sed从宽带测光得到。
我们展示了我们的公开软件hyperz,它目前在网上是可用的。
我们介绍了方法,而且我们讨论了不同的观测条件和理论假设所产生的预期影响。
通过实测数据和模拟数据,特别仔细的讨论了最小化过程中所用到的模板组合(年龄,金属丰度,红化,莱曼森林的吸收,……)的影响。
我们给出了测光红移的预期精度、灾难证认和错误判断与红移范围、所使用的滤光片、以及测光精度的关系。
 
关键字:星系:红移-普遍的--方法: 数据分析-测光
 
可以在文章中找到两种本质上不同的测光红移技术:所闻的经验训练集合方法,和用合成光谱或者经验的光谱模板去拟合观测到的光谱能量分布(以后记作SED)。
第一种方法,最初是由Connolly建议(1995,1997),在观测对象的有光谱红移的子样本中,找到星等和红移的经验关系,即训练集合。
这个方法有一个简单修改版本,Wang等人(1998)用颜色的线性方程来推断HDF-N的红移。即使可用的滤波片数量很少,这一方法也只产生很小的弥散,而且它的优点在于不对星系光谱和演化做任何假设,因此绕过了我们对高红移光谱的无知。
然而,这种方法并不灵活:当使用不同的滤波片组合的时候,星等和红移之间的经验关系要对每一个巡天重新选取合适的光谱红移子样本并重新计算。此外,训练集合是由最亮的可能测量红移的物体组成的。因此这类的过程在计算暗源红移的时候,原则上会产生一些偏差,因为从分光光度的角度来讲,没有证据表明我们在处理同一类物体。而且,目前为止,在1.4到2.2的红移区间内缺少明显的光学光谱特征,光谱红移很难测量。因此在这一区间不能找到可信的经验关系。
 
 
下一节将要讨论的SED拟合程序,它的效率取决于光谱整个形状的拟合,也取决于显著光谱特征的探测。
新的Bruzual \& Charlot 演化程序(GISSEL98,Bruzual \& Charlot 1993) 被用来构建8种不同的合成的恒星形成历史,粗略地符合了本地星系的观测性质,从椭圆星系到不规则星系:星爆系统,恒定的恒星形成系统,和6个由不同衰减时间表征以符合从椭圆星系到S0、Sd型星系颜色序列的 u模型(指数衰减SFR)。
 
我们使用由Miller \& Scalo (1979)计算的初始质量方程(IMF),但是这一选择对最终结果的影响是可以忽略的,我们会在4.6节讨论这个问题。
恒星形成的质量的上限是125个太阳质量。基本的数据包括拥有太阳金属丰度的SED,但其他的可能性将在第四节讨论。
 
这个模板库也包括了一系列经验的模板,他们由Coleman, Wu 和 Weedman(1980)(以下记作CWW)编译,来再现本地的星系。
 
CWW光谱用相应的GISSEL98 光谱扩展到波长小于1400A 和大于10000A的范围。
这个合成的光谱库是从Bruzual \& Charlot 建立的,包括408条光谱(51个不同星族的年龄记录和8中恒星形成机制)。
在大多数应用中,当u模型的数量减小到3,因此包括255光谱的时候,结果没有明显的变化。
 
观测到的测光SED要通过相同的测光系统,与一系列本地光谱相比较。给定物体的测光红移对应于一个模板SED对观测SED的最佳拟合。这个方法主要应用在HDF,利用观测或者合成光谱(即,Mobasher et al. 1996,Lanzetta et a,. 1996, Furusawa et al. 2000)。
考虑所有情况,一个决定性的测试是比较从一个限制样本中得到的测光红移和光谱红移。
Benitez(2000)提出将这种方法和引入优先概率的贝叶斯边缘化组合在一起。
 
这篇文章的目的是,直接地解释从宽带测光数据计算测光红移的表现和限制。
这项研究通过我们的公开软件hyperz进行,但是大部分结果对与这一类计算是完全适用的。
这个程序最初是由Miralles(1998)(另见Pello et al. 1999),hyperz的现行版本在下面的网站上:
http://webast.ast.obs-mip.fr/hyperz .
 
这篇文章的行文安排如下。
在第二节,我们讲述了hyperz所用的方法,包括它的一系列参数。
在第三节,通过模拟分别研究了红移测量的精度和灾难证认的比例跟滤光片、测光精度的关系。
不同参数对测光红移精度的影响将在第四节讨论,利用了模拟数据和HDF的光谱数据。
第五节用来分析处理深度测光观测的真实数据时的测光精度和可能的分类学。
第六节给出了一般性讨论,第七节列出了结论。
 
2.方法
测光红移(此后记作 zphot)基于明显光谱特征的探测,例如4000A断裂,巴尔默断裂,巴耳末减幅或者强发射线。
通常,宽带滤光片只能够探测“断裂”,而且他们对发射线的存在不敏感,除非发射线对特定滤光片的所有流量的贡献大于或者等于测光误差的的数量级,就像AGNs的情况(Hatziminaoglou et al. 2000)。
用一系列的模板光谱去比较观测到的SED:
 
其中分别是观测和模板通过滤光片i的流量和他们的误差,b是归一化常数。
 
新的Bruzual \& Charlot 演化程序(GISSEL98,Bruzual \& Charlot 1993) 被用来构建8种不同的合成的恒星形成历史,粗略地符合了本地星系的观测性质,从椭圆星系到不规则星系:星爆系统,恒定的恒星形成系统,和6个由不同衰减时间表征以符合从椭圆星系到S0、Sd型星系颜色序列的 u模型(指数衰减SFR)。
 
我们使用由Miller \& Scalo (1979)计算的初始质量方程(IMF),但是这一选择对最终结果的影响是可以忽略的,我们会在4.6节讨论这个问题。
恒星形成的质量的上限是125个太阳质量。基本的数据包括拥有太阳金属丰度的SED,但其他的可能性将在第四节讨论。
 
这个模板库也包括了一系列经验的模板,他们由Coleman, Wu 和 Weedman(1980)(以下记作CWW)编译,来再现本地的星系。
 
CWW光谱用相应的GISSEL98 光谱扩展到波长小于1400A 和大于10000A的范围。
这个合成的光谱库是从Bruzual \& Charlot 建立的,包括408条光谱(51个不同星族的年龄记录和8中恒星形成机制)。
在大多数应用中,当u模型的数量减小到3,因此包括255光谱的时候,结果没有明显的变化。
 
我们通篇使用同一组宽带滤光片,它们的相关参数如表1所示。 这些滤光片覆盖了我们所要研究的所有波长范围,没有明显的重叠和空隙。
其中也包括了HDF所用的滤光片,它们将在第四节和第五节用到。
hyperz的滤光片库是Bruzual \& Charlot滤光片库的扩充版本,目前包括163个滤光片和响应函数。文章中所有的星等都是Vega星等。
 
hyperz对大样本红移计算的效率做了特别优化。对于给定样本,输入数据是星等和测光误差。为了得到测光红移的可靠的估计,必须仔细考虑获取星系颜色和与其对应的测光误差的过程,包括测光零点引起的不确定性,内部精度等等。所有获得星等的照片必须经过不同的大气视宁度校正,滤光片必须采用统一的通光口径。对于一个给定的样本,与测光红移计算有关的参数有:
 
-- 模板光谱集合。恒星形成率的形式,可能存在于星族年龄和金属丰度之间的联系和IMF的选择。我们将在第四节讨论它。
-- 通常选择Calzetti et al. (2000),但是其他4个定理也包括在软件中。这会在4.4节讨论。输入值是Av,对应于一个尘埃屏模型,F0和Fi分别是观测到的流量和原有的流量。在波长处的消光通过公式
与色余和红化曲线相联系,除了小麦哲伦云和Calzetti的红化定理(R=4.05),上式中的R=3.1。
通常Av被设置在0到1.5星等范围内。对给定光线的平均星际消光改正可以通过色余引进,并且它将作用于整个样本。
 
-- 莱曼森林的流量减幅根据Giallongo \& Cristiani(1990)与Madau(1995),的理论计算,此二者给出相似的结果。
 
每个滤光片的极限星等,和用于没有探测到的物体的规则。这个规则独立地应用于每一个滤光片,有4种情况:
0)这个滤光片的数据不参与计算;
1)通过这个滤光片的流量为0,相应的测光误差从这个滤光片的极限星等计算;
2)通过这个滤光片的流量根据极限星等,为极限流量的一半,相应1 $\sigma $ 的误差为这一流量的一半;
3)流量和1 $\sigma $ 的误差从极限星等和极限星等的误差计算(两者都是固定的)。
当处理相对较深视场的观测时情况1是所用滤光片的一个通常设置,情况0和情况3通常用于“视场外”物体。情况2和情况3非常适合较浅视场的观测.
这里的“深”和“浅”是指测光样本中不同滤光片极限星等的相对值。
 
-- 宇宙学参数,$H_0$,$\Omega_0$和$\Omega_\Lambda$.
他们只与给定红移处的星族的最大年龄有关。年龄检查是一个可选参数。
 
恒星形成率的形式,年龄,金属丰度和红化定义了计算给定物体测光红移的参数空间,由于这一参数空间的简并,计算过程相当于在整个参数空间中寻找最相似的红移,与最佳拟合SED的细节关系不大(见图1)。
测光红移和SED,以及最佳拟合的参数(消光,光谱型,金属丰度和年龄)都来自hyperz。由于这些参数之间的简并,相关的信息应该是红移和粗略的光谱型,就是说,一个给定物体在指定红移处有一个“蓝”的或是“红”的连续谱,但是只通过快带测光不能获得其他参数的可信的信息。
 
3 滤光片和测光精度
这一节我们通过模拟来研究测光红移质量和滤光片、测光精度和红移的关系,即红移测量的稳定性和灾难证认、错误判断的比例。
这个练习的目的在于研究由SED样本和测光误差所产生的系统性影响。灾难证认(l\%)是指$|\Delta_z|=|z_{model}-z_{phot}|\geq1$的那些物体,因此这些物体不在他们原有的红移区间。
在给定红移区间,红移精度由对模型红移的中误差来定义:$\langle\Delta_z\rangle=\Sigma\Delta_z/N$.
其中不包括灾难证认,标准差这样定义:$\sigma_z=\sqrt{\Sigma(\Delta_z-\langle\Delta_z\rangle)^2/(N-1)}$.
伪证认(g\%)对应于那些被错误地归于给定测光红移区间的物体,因此他们容易污染了这一红移区间的统计性质,在这种情况下,
$|\Delta_z|\geq3\times\sigma_z$.
 
他们中的有些量,尤其是l\% 和g\%,他们依赖于红移计数的假设和测光的深度。正因为这样,我们只对更真实的星系计数模型做模拟计算,这个模型基于纯光度演化理论。在第五节,我们讨论这些结果和测光参数之间的关系。
 
为了计算上面提到的参数与滤光片和测光精度的关系,我们产生了一个用来做模拟的样本,包括1000个物体,均匀的红移分布。
在所有情况下,不同红移区间中的不同星系的光谱型和年龄是从前面提到的8个GISSEL98模板中随机选取的,他们有同样的太阳金属丰度。
 
这一均匀样本的测光误差是作为一个高斯型的噪声引进的,在每一个波段上都是固定的1$\sigma$(0.05到0.3个星等,即~5\%到30\%的测光精度),并且对于不同的滤光片它们是不相关的。
对于每个滤光片组合,我们考察测光红移的质量与测光精度的关系。
对于这种情况,测光误差不以星等为单位。
第五节用到了一个真实的误差分布。
可见光波段的消光值在0到1的范围内。
对每个模拟星系,hyperz计算一个测光红移值,同时也计算相应概率为P=68,90,99\%自信度区间的测光红移误差棒,它是用$\delta_{\chi^2}$增加的方法对一个参数计算出来的(Avni 1976)。
在z=0和z=7之间,用来搜索结果的红移步长是$\delta_{\chi^2}$=0.05,而且有一个内部精度是这个值的十倍。0.1到0.05之间选择主要的红移步长都不会明显的影响结果.
 
当测光红移去和红移的真实值模型红移比较的时候,图2显示了不同模拟样本的情况。这些模拟的结果被列在表2中。
正如我们所预期的,缺少近红外测光数据时,单个星系的误差在1.2<=z<=2.2之间变得很大,应为在可见光波段缺少明显的光谱特征。在这个红移区间,4000A减幅超过i波段,而莱曼减幅还没有影响到u波段的测光。
当近红外测光数据被包括进来,这一问题便得到了解决。事实上,J、H和K波段的滤光片能够包含4000A减幅。
另外,在z<=0.4的红移区间,u波段测光数据的缺失加强了测光红移结果的不确定性(主要是由于z<=0.2区间的贡献),因为在这一区间没有其他的滤光片能够探测到一个强的减幅。
 
所有这些结果几乎跟星系形态无关,只要星族的年龄超过$10^7$年这个典型年龄。这一点将在下一节仔细讨论。
 
测光红移的误差对测光不确定性非常敏感。$\Delta_m$ $\leq$0.05,并没有明显的好处。
若考虑到所有误差源,这一数值粗略地相当于深空测光观测的典型精度。
多重解的弥散和数量很快地增大,直到 $\Delta_m$ $\leq$0.3。
引进近红外JHK测光大幅减小了1.2<=z<=2.2红移范围内的误差棒,而在这个范围以外的测光红移的精度并没有明显的提升。
如果在五个光学波段上再加上Z波段,在低红移处结果的弥散将变小,直到$z_{model}~1.5$,但是$z_{model}$=1.5-3之间的简并仍然存在,
 
图3展示了两个分别在低红移和高红移处的模拟星系的概率分布方程:当测光精度增加,波长范围扩展到近红外时,计算结果更好地趋近于模型的真值,而且高低红移解之间的简并也消失了。
 
这里得到的测光红移的弥散跟文章中找到的值是相似的,即使所用的技术是截然不同的。
在大多数发表的研究中,很难比较测光精度对红移结果精度的影响。
 
对于一个给定的滤光片和测光精度,这些结果有利于我们理解结果的整体趋势。
然而,测光红移技术经常用来做统计研究,他们需要更真实的模拟来为测光观测作出合理的观测刚要。
这样,就需要有一个更真实的红移分布。对大多数的应用,一个PLE模型已经足够决定主要的趋势。而且,为了再现真实样本的行为,测光不确定性必须以星等为单位。这些问题在第五节讨论。
 
4.1 模板和莱曼森林
 
对HDF样本我们比较了hyperz的测光红移测定结果和真实的光谱红移数据,研究了模板组合对最终结果的影响。
我们把所有其他的参数固定,唯一的不同是用来计算测光红移的滤光片组合。
表3总结了这些结果。最近Hoog以不同的方式,特别是不同的滤光片组合,对HDF-N星系的$z\leq1.4$样本做了类似的测试。
 
我们对HDF-N中有光谱红移的108个样本计算了测光红移,这个样本来自Fernandez-Soto加上4个来自HDF-S的星系。
它们当中,有83个在红移小于1.5范围内,另外29个在2 $\le$ z $\le$ 6。
测光数据来自Stony Brook的小组,他们用SExtractor去探测目标,并且考虑了HDF-N的7个滤光片(F300W,F450W,F606W,F814W加上Dickinson等人1998年在KPNO IRIM望远镜做的近红外JHK波段的测光数据),和12个HDF-S滤光片(F300W,F450W,F606W,F814W,加上来自NTT SUSI2的浅的光学UBVRI样本以及从NTT SOFI获得的近红外JHK数据)。
这里,我们对HDF-N样本应用7个滤光片,对4个HDF-S星系应用所有可用的12个滤光片。
用7个滤光片去计算HDF-S样本对单个星系的测光红移和整体的统计性质没有明显的影响。
 
为了从样本中可用的观测流量计算星等,我们采用这样的标准来判断为探测物体:S/N<1。
这样,我们把这个值记为99,并且应用所用滤光片的限制星等。
 
在这一节,我们考虑了三个不同的模板组合:基本的5个GISSEL98模型,太阳金属丰度
 
上文提到的太阳金属丰度的5个基本GISSEL98模型(1个星爆模型,3个衰减模型,1个恒定恒星形成率的系统),CWW的经验SED组合和用GISSEL模板库(Miller \& Scalo IMF,恒定的恒星形成率,年龄为0.1Gyr)中极蓝星系SED扩展的CWW模板。
为这三个模板组合再增加一个新的极蓝星系光谱不会明显地改变结果。
对于模拟样本我们在z=0 - 7的范围内求解,步长为$\delta_z$=0.05。
对所有情况,我们对决定星等增加了一个粗略的限制:$M_B$[-18,-9]。
此外,我们检查了模板的年龄,确保这一年龄小于星系所在红移处的宇宙年龄,这取决于宇宙学模型。
这里我们采用这样的参数:
假设红化在Av=0到1.2范围内,满足Calzetti等人(2000)的定律。
 
 
 
 
图4是这112个星系的光谱红移和测光红移的比较,这三个不同的模板组合此后分布记作(a),(b)和(c)。
每一组都与光谱红移有很好的一致性,但是考虑弥散时,他们出现了明显的不同,在这两个红移区间($z<1.5和2<z<6$)这样计算弥散:
 
-在$z_{spec}< 1.5$这个红移区间,我们发现:
(a)=0.09;(b)0.21;(c)0.17。如果我们除去的物体,减小到(a)0.06(81个物体);(b)(81个物体)和(c)(82个物体)都是0.08。
在情况(a), 被剔除的物体有着不确定的光谱红移。
因此在这个红移区间,用GISSEL98模板的得到的红移精度要好于只用CWW模板得到的结果。
 
-- 在高红移区间,所有情况的弥散都为
如果我们去掉由定义的灾难证认(2个物体),则结果是这样:(a),(b),0.08;(c),0.07。
这样CWW得到的结果略好于GISSEL,可能是由于莱曼覆盖效应。这一点在下面讨论。
 
通常这样的结果可以归因于很多原因,比如错误的测光(测量星等的系统误差或者过小估计的测光误差)所导致的很不相像的拟合,或者拟合参数之间的简并会导致概率函数出现很突出的第二级大,或者由于缺少有效的测光信息概率函数显得很“平”。
最后一种解释恰好适合光谱红移在5.64的这个物体,它在F450W的探测极限上,只能在F814W滤光片中看到,信噪比大约是1.5。然而,如果我们采用所有可用的测光数据,忽略信噪比标准,我们得到的测光红移是5.13。光谱红移在2.931的物体被其他的小组证认为低红移。虽然如此,在红移为2.90处找到一个对应很小概率的第二级大。
 
我们注意到在高红移,情况(b)和(c)的结果与光谱红移符合的更好。然而,$ \chi^2$要比(a)大。可能的原因是这里引入的莱曼森林和红移的一对一关系。为了研究这个问题,我们指定了不同的莱曼减幅的值,再把线覆盖的平均值乘上0.5和1.5,
这两个值由Madau(1995)给出的,然后增大或者减小吸收。
当莱曼森林产生比平均值小的吸收时,我们对HDF数据得到了一个更好的符合。
这时,我们得到:GISSEL(a),一个跟CWW SED相似的值。对中性氢吸收的过高估计导致了对红移系统性的低估,因为同样的消光可以用更低红移的解来再现。所以,为了得到精确的测光红移,对UV波段SED的了解是极其必要的;而且,莱曼森林代表着高红移光谱最重要的特征。因此,允许莱曼森林的吸收达到一个很宽的范围才能避免对高红移产生系统性的影响,这个范围由光线来决定。
 
有必要指出,即使所有的模板SED很精确地再现了HDF样本的光谱红移,当我们所研究的样本超过光谱红移极限时,星系红移的分布会产生很大的变化,而且这时没有可用的训练样本。比较HDF样本用CWW模板得到的红移分布和用GISSEL模板得到的红移分布,整体上它们之间没有明显的差别。然而,这样的结果不适用于所有的情况。一个直接的例子就是没有近红外,只有光学波段的深空测光观测,这种观测用来观测低表面亮度区域。容易看到,在这种情况对最暗的“蓝”源会出现简并解,不可能分辨它是一个低红移解(低表面亮度的非常年轻的星族),还是一个恒定恒星形成的相对较亮的$1<z< 2.5$星系(连续谱的蓝端没有明显特征)。
 
这时,只用CWW模板趋向于系统性地选择后者,然而使用具有很宽的星族年龄的模板(就像GISSEL模板)会选择前者,因此这导致了完全不同的红移分布。我们倾向于选择数量更多的GISSEL模板,它很好地支持不同年龄的影响,而不是通过不同光谱型的本地星系来假设星系的演化。
 
4.2 星族的年龄
 
当通过滤光片所探测到的光谱特比测光误差更强的时候,测光红移才是有效的。当我们处理一个年轻星族的连续谱时,4000A减缩到大约$10^7$年才能观测到。大多数情况下,这种光谱特征的缺失不能由强发射线的存在所弥补,只是因为当使用宽带测光时,这些发射线对积分能力的影响可以忽略。
 
为了研究年龄对红移估计的影响与红移的关系,我们产生了不同年龄的样本,所有样本都有一致的红移分布,都是星爆SED(单星族模型)。
图6,显示了测光红移与模型红移之间对比的整体趋势,这些样本有不同的年龄,使用的是UBVRIJHK滤光片。
 
 
这里使用的是基本的GISSEL太阳金属丰度的模板。
在$z_{model}\geq3$,由于莱曼减缩可以在U波段观测到,测光红移的测定对任何年龄都是精确的。在更小的红移处,测光红移取决于$4000\AA$减缩,这时它是最突出的光谱特征,但是它只能在年龄达到几个$10^7$年的系统中才能看到。
 
对这些样本,图5画出了hyperz所给出的结果,图中我们用4个红移区间的弥散展示了前文所讨论的效应:增大红移和星族年龄,的值将减小。
 
4.3 宇宙学
 
宇宙学参数($H_0$,$\Omega_0$和$\Omega_\Lambda$)的影响只跟指定红移处所允许的星族年龄有关。使用hyperz时,可以选择是否把星族的年龄限制在宇宙学参数允许范围内。
为了定量描述对测光红移结果的影响,如果存在的话,我们比较了有这种粗略的年龄限制的结果和没有年龄限制的结果,而且我们使用了不同的宇宙学参数($H_0$,$\Omega_0$和$\Omega_\Lambda$)。
结果显示,宇宙学参数对测光红移结果的影响可以忽略,因为它们对$\delta_z$的影响小于1\%。
 
 
 
 
 
4.4 红化
 
目前hyperz采用的5个红化定律如下:
 
1.Allen(1976)银河系(MW)红化定律;
2.Seaton(1979)红化定律,被Fitzpatrick(1986)用于银河系;
3.fitzpatrick(1986)大麦哲伦云(LMC)红化定律;
4.Prevot 等人(1984)和Bouchet等人的小麦哲伦云(SMC)红化定律;
5.Calzetti等人(2000)星爆星系红化定律。
图7画出了不同的定律。
 
最近的高红移星系研究和被尘埃掩食的恒星形成研究表明了红化对高红移宇宙的重要性。
为了研究这一问题对测光红移计算的影响,我们比较了对HDF样本应用红化得到的结果和没有红化时的结果,所有其他的参数都是固定的。我们得到,对低红移区间$\delta_z=0.13$(去掉灾难证认物体之后为0.07),对高红移区间$\delta_z=0.50$(0.13),但是在这一区间有很高的灾难证认比例:10个$z_{spec} \simeq 3$ 物体被错误的证认为低红移星系。
 
因此,保持一个宽的红化范围对重现高红移星系SED是必要的。
根据Steidel等人的工作,
z~4的星系的典型的色余$E_{B-V}$是0.15mags,因此用Calzetti红化定律时候是$A_v\simeq0.6 mags$。我们的计算允许的最大$A_v$大约是这个值的2倍。
 
另外,我们通过测试所有的可能性研究了不同红化定律的影响。我们发现,再现银河系和大麦哲伦云的消光的定律不能很好的拟合高红移($z_{spec}>2$)星系的SED,然而它们对低红移区间没有影响。另一方面,第四个与小麦哲伦云对应的红化定律,得到了与Calzetti等人(2000)的定律相似的结果。它正确地判断了高红移物体的测光红移,但是它把更高的$z_{phot}$赋予了一对低$z_{spec}$物体。最后的效应可能是由于与Calzetti相比,这一很好定律的$k(\lambda)$在短波段上更高和更陡,导致过多地剪掉了莱曼森林导致的UV消光。
在高红移处,最重要的波段是UV,在$1000\AA$和$3000\AA$之间,这些红化定律给出了非常不同的趋势,因此它们以不同的方式改变了星等,导致了不同的$\chi^2$值。
事实上,对HDF样本应用从1到4的红化定律所作的大多数拟合都产生了比应用Calzetti的定律逐渐变坏的结果,特别是对于那些需要$A_v>0.6$的物体。
这些星系不能被MW和LMC的红化定律再现,即使$A_v$的限制增加到$A_v=2$。
 
因此,所选择的红化定律在短波段的斜率需要仔细的定义;
这里,为了拓展1到4红化定律的波长范围所作的外推是非常差的。为了正确地对高红移星系的SED建立模型,UV波段的SED模型是必要的,这种必要性是上述考虑的更有力的证明。
尘埃被恒星形成区加热而发生再辐射,这种辐射能量不会影响现有的结果,因为我们关心的是UV和近红外波段。
 
4.5 金属丰度
 
我们也用同样的HDF样本测试了金属丰度对$z_{phot}$估计的影响。
我们用不同的和极端的星族金属丰度假设做了同样的计算,取值范围从$0.005Z_\odot$到$5Z_\odot$。
我们也制作了一个自洽的模板集合,其中仔细地考虑了星族金属丰度演化(cf. Mobasher \& Mazzei 1999)。也就是说,星族的年龄和平均金属丰度有自然的联系。对所有的金属丰度情况,我们建立了与前面一样的闭合系统:恒定的恒星形成和6个$\mu$模型。
 
我们考虑了3个不同的模板组合:同时考虑3个不同的金属丰度(太阳金属丰度和2个极端值),只有两个极端值和自洽的模型。
这些情况的比较画在图4(d,e,f)中。对三种不同情况,低红移处没有证认失败的物体,弥散分别是$\delta_z=0.05$,0.06,0.05。对于同样的假设,在高红移处,我们得到$\delta_z=0.11$,0.10,0.10。
当同时使用几种不同的金属丰度时,在$z\leq1.5$观测到$z_{phot}$的精度有微小的提高,而且,在这个红移区间自洽模型拟合的最好。
在另一方面,包含不同的金属丰度对高红移证认没有影响。
 
4.6 初始质量函数
 
我们用HDF光谱样本研究了初始质量函数的影响。
我们用自洽的模型,这个模型考虑了星族金属丰度的演化,使用Miller \& Scalo IMF(1979)与HDF数据符合的最好。
我们也建立了2个同样的闭合模型,用两个另外的IMF,Salpeter(1955),和Scalo(1986),使用同样的恒星形成质量上限。当我们对HDF样本应用这些新的模型,我们得到了非常一致的红移精度。更仔细地分析对每个物体得到的结果,我们发现不论使用哪一个IMF,它们对$z_{phot}$的估计几乎是一样的。这一结果很容易理解,因为不同的IMF对恒星连续谱造成的差异大都会被其他的参数(红化,年龄,$\dots$)弥补,以此,它们给出相同的测光红移结果,参数空间中不同的解。
 
当我们对模拟数据计算$z_{phot}$,得到相同的$z_{phot}$精度,不论我们在模型星系和SED模板中使用一致的IMF或者不同的IMF。另外,我们检查了后一种情况下,hyperz推算光谱型是产生系统性改变的可能性,得到了可以忽略的结果。
特别是,用Salpeter和Scalo IMF去分析由Miller \$ Scalo IMF建立的模型样本,结果与4.8节相同。这使我们更加相信IMF在$z_{phot}$估计中只是次要的参数。
 
4.7 发射线
 
只要我们在处理宽带测光数据,光谱发射线的存在对积分流量只有相对小的影响,因此对$z_{phot}$结果有很小的影响。当我们考虑Guzm$\acute{a}$n等人研究的在$z\leq1.4$的蓝致密星系,和Cowie等(1995),Glazebrook等(1995),Terlevich等(1991)研究的恒星形成星系样本,这一点可以很容易地量化。
在相对较低的红移处,主要考虑的发射线是$[O_{\uppercase\expandafter{\romannumeral2}}]\lambda3727$,$H_\alpha$,$H_\beta$,和$[O_{\uppercase\expandafter{\romannumeral3}}]\lambda4959$,5007,其中$[O_{\uppercase\expandafter{\romannumeral2}]}$和$H_\alpha$对积分流量的贡献最大。
根据Guzm$\acute{a}$n等人(1997)的研究,对于恒星形成星系,$[O_{\uppercase\expandafter{\romannumeral2}}]\lambda3727$光度可以近似的写成$L([O_{\uppercase\expandafter{\romannumeral2}}])~10^{29} W_{[O_{\uppercase\expandafter{\romannumeral2}}]}L_B$,其中,$W_{[O_{\uppercase\expandafter{\romannumeral2}}]}$等值宽度,$L_B$是以太阳为单位的蓝波段光度。
 
考虑我们的目的,当$f(e\textendash line)/f_\lambda\leq1-10^{-0.4\Delta_m}$时,我们可以忽略发射线,其中$f(e-line)$和$f_\lambda$分别是发射线中的积分流量和通过滤光片的恒星连续谱,$\Delta_m$是以星等表示的测光不确定性。
相对值$\Delta_m~0.05$到0.1星等(~5到10\%不确定性)对应$f(e\textendash line)/f_\lambda\leq0.05$到0.1。
对于Guzm$\acute{a}$n等人样本中的星系,等值宽度的极限是100$\AA$的几倍,因此大多数致密的恒星形成星系完全满足这个条件。即使我们考虑激烈恒星形成源的典型光度,发射线对大多数源的影响依旧可以忽略。而且Terlevich等人(1991)本地样本中的大量$H_{\uppercase\expandafter{\romannumeral2}}$星系也完全满足这个条件。
 
因此,发射线似乎对恒星形成星系的$z_{phot}$结果没有显著的影响。
相反,当我们研究AGN,或者使用窄带滤光片获得测光数据时,这个结论就不适用了。这里我们没有考虑AGN对模拟样本的影响,也没有考虑当处理真实数据时,这种SED模板对最后精度的影响。
在我们的方案中很容易引进AGN的SED,而且这种特殊的应用正在发展中(Hatziminaoglou et al. 2000)。
 
$\mathbf{djfief}$%使用粗体
$\romannumeral20$
$\uppercase\expandafter{\romannumeral20}$
 
\end{document}