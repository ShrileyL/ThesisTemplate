% XeLaTeX can use any Mac OS X font. See the setromanfont command below.
% Input to XeLaTeX is full Unicode, so Unicode characters can be typed directly into the source.

% The next lines tell TeXShop to typeset with xelatex, and to open and save the source with Unicode encoding.

%!TEX TS-program = xelatex
%!TEX encoding = UTF-8 Unicode

\documentclass{beamer}
\usepackage{geometry}                % See geometry.pdf to learn the layout options. There are lots.
%\geometry{landscape}                % Activate for for rotated page geometry
%\usepackage[parfill]{parskip}    % Activate to begin paragraphs with an empty line rather than an indent
\usepackage{graphicx}
\usepackage{amssymb}

%\usetheme{Warsaw}
\usetheme{AnnArbor}
%\usetheme{PaloAlto}
%\usetheme{Antibes}
%\usetheme{CambridgeUS}
%\usetheme{Copenhagen}
%\usetheme{Darmstadt}
%\usetheme{Dresden}
%\usetheme{Frankfurt}
%\usetheme{Goettingen}
%\usetheme{Hannover}
%\usetheme{Ilmenau}
%\usetheme{JuanLesPins}
%\usetheme{Luebeck}
%\usetheme{Madrid}
%\usetheme{Marburg}
%\usetheme{PaloAlto}
%\usetheme{Singapore}
%\usetheme{Warsaw}

% Will Robertson's fontspec.sty can be used to simplify font choices.
% To experiment, open /Applications/Font Book to examine the fonts provided on Mac OS X,
% and change "Hoefler Text" to any of these choices.

\usepackage{fontspec,xltxtra,xunicode}
\defaultfontfeatures{Mapping=tex-text}

\setromanfont[Mapping=tex-text]{STKaiti} %设置中文字体
\setsansfont[Scale=MatchLowercase,Mapping=tex-text]{STKaiti} 
\setmonofont[Scale=MatchLowercase]{STKaiti} 

%\setromanfont[Mapping=tex-text]{Hoefler Text}
%\setsansfont[Scale=MatchLowercase,Mapping=tex-text]{Gill Sans}
%\setmonofont[Scale=MatchLowercase]{Andale Mono}


\newfontfamily{\H}{STHeiti}
\newfontfamily{\K}{STKaiti}
\newfontfamily{\E}{Arial}
 \newfontfamily{\S}{STSong}
 


\title{PPT Template}
\author{Shirley Liu}
\date{\today}                                           % Activate to display a given date or no date

\begin{document}

%%%title frame
%\begin{frame}
%	\titlepage
%\end{frame}

\AtBeginSection[]{
\begin{frame}{Table of Contents}
	\tableofcontents[currentsection]
\end{frame}
}

\begin{frame}
	\thispagestyle{empty}
	\titlepage
\end{frame}
\addtocounter{framenumber}{-1}

\begin{frame}{Table of Contents}
	\tableofcontents
\end{frame}

%%%%%%%%%%%%%%%%%%%%%%%%%%%%%%%%%%%%%%%%
\section{Section One}

%%%%%%%%%%%%%%%%%%%%%%%%
\subsection{SubSection One}

\begin{frame}{What Is Beamer?}
	\begin{itemize}
		\item Beamer is a flexible \LaTeX\ class for making slides and presentations.
		\item It supports functionality for making PDF slides complete with colors, overlays, environments, themes, transitions, etc.
		\item Adds a couple new features to the commands you've been working with.
		\pause
		\item As you probably guessed, this presentation was made using the Beamer class.
	\end{itemize}
\end{frame}

%%%%%%%%%%%%%%%%%%%%%%%%

%%%%%%%%%%%%%%%%%%%%%%%%
\subsection[Basic Structure]{SubSection Two}

\begin{frame}[fragile]{Document Template: slides.tex}
\begin{columns}[t]
  \column{0.5\textwidth}
  \begin{block}{}
    \begin{verbatim}
\documentclass[pdf]
          {beamer}
\mode<presentation>{}
%% preamble
\title{The title}
\subtitle{The subtitle}
\author{your name}

\begin{document}
  	\end{verbatim}	
  \end{block}
  
	\column{0.5\textwidth}
	\begin{block}{}
		\begin{semiverbatim}
\%\% title frame
\\begin\{frame\}
    \\titlepage
\\end\{frame\}

\%\% normal frame
\\begin\{frame\}\{Frame title\}
    The body of the frame.
\\end\{frame\}

\\end\{document\}
		\end{semiverbatim}
	\end{block}
\end{columns}
\verb+athena% make slides.pdf+
\end{frame}

\begin{frame}[fragile]{What would you like in your sandwich?}
\begin{itemize}
	\item So what can you do between \verb=\begin{frame}= and \verb=\end{frame}=?
	\pause
	\item Pretty much anything you can do in a normal \LaTeX\
	document:
	\pause
	\begin{itemize}
		\item figures, tables, equations, normal text, etc.
	\end{itemize}
\end{itemize}
\end{frame}

\begin{frame}[fragile]{Too much \LaTeX\ for your frames to handle?}
	\begin{itemize}
		\item Be careful not to fit too much text, especially when writing mathematical formulas, but if you must...
		\pause
		\item Use the optional argument \verb=[allowframebreaks]= with \verb=\begin{frame}= to let \LaTeX\ automatically segment your very long frame.
		\pause
		\item Warning: Overlays are not available with \verb=[allowframebreaks]=.
	\end{itemize}
\end{frame}

\begin{frame}{Don't Do This}
\begin{itemize}
	\item Here is a well-known formula:\vspace{-.5em}
	$$\displaystyle \sum_{k=0}^{n} k = \frac{n(n+1)}{2}$$
	\item Here is a less well-known, but still useful, formula:\vspace{-.5em}
	$$\displaystyle \sum_{k=0}^{n} k^2 = \frac{n(n+1)(2n+1)}{6}$$
	\item This is pretty well-known, too:\vspace{-.5em}
	$$\displaystyle \sum_{k=0}^{n} k^3 = \left(\frac{n(n+1)}{2}\right)^2$$
	\item Who knows about this one?\vspace{-.5em}
	$$\displaystyle \sum_{k=0}^{n} k^4 = \frac{n(6n^4 + 15n^3 + 10n^2 + 1)}{30}$$
	\item Have fun factoring the quartic expression!
\end{itemize}
\end{frame}

%%%%%%%%%%%%%%%%%%%%%%%%%%%%%%%%%%%%%%%%

%%%%%%%%%%%%%%%%%%%%%%%%%%%%%%%%%%%%%%%%
\section{Section Two}

%%%%%%%%%%%%%%%%%%%%%%%%
\subsection{SubSection One}

\begin{frame}
\frametitle{Page One}
Here are some multilingual Unicode fonts: 楷体
\begin{itemize}
 \item this is Hebrew: {\H 黑体}, 
   \item and here's some Japanese
   \item here's Chinese: {\S 宋体}
\end{itemize}
\end{frame}



\end{document}  